\chapter{Introduction}
	\label{CH_Intro}

Accurately and reliably predicting 3D structures, from protein sequences 
is one of the most challenging tasks in computational biology, and has been of great
interest in bioinformatics. An important intermediate step of predicting the whole 3D structure 
is correctly predicting the secondary structure of a protein\cite{yaseen2014context}. In recent years, 
Deep neural networks have been widely apply to the problem of protein secondary structure prediction and continuously pushing the state-of-the-art forward. For example, in\cite{zhou2014deep} a convolutional Generative Stochastic Network achieved 66:4\% Q8(8 categories) on CB513 dataset.  In \cite{Z.Li2016} a convolutional recurrent network which is a combination of convolutional network and recurrent network achieved 69.7\% Q8 accuracy. And in \cite{busia2016protein} they use a multi-scale convolutional network together with many techniques that help accelerating training and prevent over-fitting, such as dropout\cite{srivastava2014dropout}, batch-normalization\cite{ioffe2015batch} and regularization. The best Q8 result they report on CB513 is 70\%. 

One advantage of deep learning and deep neural network is the reusability. On one hand, a certain kind of architecture can apply to different problems without much modification. For example, a convolutional recurrent network similar to the one in \cite{Z.Li2016} have also been apply to video activity recognition and Video Description\cite{donahue2015long}. And the convolutional architecture in\cite{busia2016protein} are originally used in image recognition. On the other hand, most complex networks contain reusable modules such as different layers, regularizer and optimizer. So in this case a good way to build a deep neural network is using a reliable deep learning framework such as Torch, TensorFlow Caffe and Theano. TensorFlow\cite{abadi2016tensorflow} is a python deep learning framwork create by google. Using TensorFlow, it easy for researchers to visualize the graph and training process. And the API designs make researchers’ code shareable, standardize how software engineers approach deep learning.

In this paper I will show how to build different deep neural networks using TensorFlow to solve the protein secondary structure prediction problem. And compare the performance of different networks and different modules. And I will particular focus on how to build a recurrent network and the technique difficulties in doing this. In charpter 2 I will introduce the background knowledge and theories. And chapter 3 will describe the models in detail. Chapter 3 will cover the implementation and technique difficulties. Chapter 4 will show the result of all the networks.
